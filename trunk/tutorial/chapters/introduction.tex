%
% chapter introduction
%

This document is meant as a tutorial for the computer vision library \textbf{ICL} (\textbf{I}mage \textbf{C}omponent \textbf{L}ibrary). It will explain different aspects, that are useful to work with the ICL:

\begin{enumerate}
\item \textbf{What is the ICL:}\\ Here the underlying goals and ideas of the ICL will be proposed.
\item \textbf{Feature Overview:}\\ The ICL provides a large list of features, packages algorithms and utility classes. To motivate the reader for reading also the following chapters, the \emph{highlights} of the ICL are presented here.
\item \textbf{What is an Image:}\\ An image representation is fundamentally for the development of high performance computer vision algorihtms. It will be explained how the image representation is implemented in C++ as base class for ICL images. In particular it will be shown why we conduct template classes in combination with inheritance for the image class and, what problems arise therewith. 
\item \textbf{The Image class:}\\ Some functionalities are implemented as member functions, however some other are not. Here the ICL's image class interface is discussed.
\item \textbf{C++ Templates:}\\ To make later chapters of the document easier to understand, C++ template classes are explained in detail. In particular, it will be shown, how templates are translated by the compiler, and how they can be used \small{\textbf{i)}} To reduce the amount of redundant source code and \small{\textbf{ii)}} To accelerate code without using constants for each possible parameter of functions.    
\item \textbf{Simple Image processing:}\\ How can simple Image processing algorithms be implemented (and how can they implemented elegantly) using the ICL.
\item \textbf{Overview over ICL Packages:}\\ The ICL consists of a set of (currently) 10 sub-packages that more [aufeinander aufbauen] on each other. In this part, the contents and the basic ideas of these packages are introduced and discussed. 
\item \textbf{Common ICL-classes and ICL-Interfaces:}\\ In many packages interfaces (such as e.g. a \inlinecode{Grabber} for images -- a image source) are defined that influenced the design of the whole library heavily. These interfaces and also a variety of other very common classes and class sets are presented and explained.
\item \textbf{Graphical User Interfaces (GUI):}\\ The ICLQt package provides a powerful wrapper for Qt-based applications. In this part it will be shown, how simple and even more complex GUI's can be created and especially, how user input can be synchronized with the applications worker thread.
\item \textbf{Writing advanced applications:}\\ Here some advanced programming techniques are proposed: Writing applications with GUI support managing several threads, handling program arguments and much more.
\item \textbf{ICL-Development:}\\ICL developers need some deeper insights into the file structure and the makefile system. These information will be given in this part.
\item \textbf{ICL-Projects:}\\Development of ICL-bases applications can be performed very conveniently using a special makefile system and directory structure provided by the so called \emph{ICLProjects} svn branch. How to get these projects, how to add own projects here and how to include other projects and external libraries is shown in this (currently) last part. 
\end{enumerate} 
