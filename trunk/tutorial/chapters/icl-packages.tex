Chapter \ref{cha:features} listed basic features of the ICL packages that do not pertain to image processing in the proper meaning of the word. This chapter lists and explains most image processing functionalities and interfaces.

\section{ICLUtils}

[Documentation: \iclpackageref{Utils}]\\
This ICLUtils package contains (per definition) no image processing functions, rather it provides more general classes and functions (e.g. matrix and vector classes, exceptions, debugging macros etc.).

\subsection{Matrix and Vector Classes}
Matrix and Vector algebra is very common in a lot of applications. The ICL provides two approaches for matrix and vector classes:


\subsubsection{FixedMatrix Template Class\iclclassref{FixedMatrix}{Utils}}
This class uses three template parameters \inlinecode{template<class T, unsigned int COLS, unsigned} \inlinecode{int ROWS>} to represent matrices of fixed size. FixedVectors are implemented as shallow inheriting wrappers of FixedMatrix class instances with \inlinecode{COLS=1} (\inlinecode{FixedColVector}) or \inlinecode{ROWS=1} (\inlinecode{FixedRowVector})\footnote{This cannot be implemented using typedefs, because C++ does not support templated typedefs}.
By using \emph{templated} dimension parameters, matrix dimension is fixed at compilation time, which enables the compiler to apply a lot of optimizations. Most common vector and matrix algebra functionalities \footnote{As far as I can see, actually a super-set of BLAS-level 3}, including matrix mutliplication (scalar product is just a special case here), QR-Matrix decomposition and matrix inverse and pseudo-inverse\footnote{QR-decomposition-based} implementation. \inlinecode{FixedMatrix} instances resign smart data handling to allow to compiler to apply special optimizations on fixed arrays. In particular, temporary instances (and their data) can be allocated on the stack, rather than having to allocate data on the heap.
The \inlinecode{FixedMatrix} class provides row- and column iterators, as well as extraction functions for rows, columns or rectangular sub-matrices.
Here's a concise example:
\codefile{fixed-matrix-demo-1.cpp}{Example for usage of the \inlinecode{FixedMatrix}-template class}

\subsubsection{DynMatrix Template Class\iclclassref{DynMatrix}{Utils}}
\todo{describe}

\todo{further parts of Utils-package}



\section{ICLCore}

Documentation: \iclpackageref{Core}\\
The Core-package contains, apart from the image base class \inlinecode{ImgBase} and template classes \inlinecode{Img<T>}, fundamental image processing functions and utility classes. In particular, it contains the header \iclheaderref{Core}{Types}, that declares most common ICL data types and enumerations. Furthermore, there are mathematic functions, random number generators, utility functions for converting and copying images and some utility structures like rectangles, points and lines.

\subsection{Image classes}

\subsection{Types}

\subsection{Mathematic Functions}

\subsection{Random Number Generators}


