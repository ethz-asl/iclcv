\newcommand{\hrefn}[1]{\href{#1}{#1}}

% Macht richtige Anfuehrungszeichen: ,, Bla ''
\newcommand{\myquote}[1]{\glqq{}#1\grqq{}}

\newcommand{\todo}[1]{%
  $\bigstar$%
  \marginpar{%
  \begin{flushleft}%
  \vspace{-1.5\baselineskip}%
  $\bigstar$\textbf{TODO:\\}%
  #1
  \end{flushleft}%
  }%
}%

\newcommand{\stress}[1]{\emph{#1}}

%buzzwords am Rand notieren
\newcommand{\buzzwords}[1]{%
  \marginpar{%
  \begin{flushleft}%
  \textbf{{\small #1}}%
  \end{flushleft}%
  }%
}%


\newcommand{\reference}[1]{%
  {(\small{$\Rightarrow$})}
  \marginpar{%
  \begin{flushleft}%
    {\small{ $\Rightarrow$ #1}}%
  \end{flushleft}%
  }%
}%
\newcommand{\mylabelledfig}[4]{
\begin{figure}[!ht]
  \centering
  \includegraphics[#4]{#1}
  \caption{
    #3
    %\picfilename{#1}
  }
  \label{#2}
\end{figure}
}

\newcommand{\includechapter}[2]{\chapter{#1\label{sec:#2}}\input{chapters/#2}\pagebreak}

%\newcommand{\inlinecode}[1]{\textbf{\texttt{#1}}}
\newcommand{\inlinecode}[1]{\lstinline[breaklines=false]^#1^}

\newcommand{\displaycode}[1]{
\begin{lstlisting}[numbers=none]^^J
#1^^J
\end{lstlisting}
}

%\def \displaycode [#1] {\\%
%begin{lstlisting}\\%
%#1\\%
%\end{lstlisting}\\%
%}

\newcommand{\cstring}[1]{\textbf{\texttt{#1}}}
\newcommand{\filename}[1]{\textbf{\texttt{#1}}}
  
\newcommand{\buzz}[1]{\buzzwords{#1}}

% F"ur die Darstellung von mathm. R"aumen
\newcommand{\mathr}[1]{\ensuremath{I\hspace{-0.39em}#1}\xspace}
% O(n) Notation
% Øexistiert eigentlich schon und produziert
% sowas wie ein durchschittssymbol
%\renewcommandØ[1]{\textit{O}\ensuremath{(#1)}}

% Abk"urzungen
\newcommand{\zB}{z.\,B.\xspace}
\newcommand{\zZ}{z.\,Z.\xspace}
\newcommand{\ZZ}{Z.\,Z.\xspace}
\newcommand{\bspw}{bspw.\xspace}
\newcommand{\bsp}{Bsp.\xspace}
\newcommand{\sog}{sog.\xspace}
\newcommand{\usw}{usw.\xspace}
\newcommand{\bzw}{bzw.\xspace}
\renewcommand{\dh}{d.\,h.\xspace}
\newcommand{\iAllg}{i.\,Allg.\xspace}
\newcommand{\idR}{i.\,d.\,R.\xspace}
\newcommand{\evtl}{evtl.\xspace}
\newcommand{\etc}{etc.\xspace}
\newcommand{\uU}{u.\,U.\xspace}
\newcommand{\uA}{u.\,A.\xspace}
\newcommand{\ggf}{ggf.\xspace}
\newcommand{\bzgl}{bzgl.\xspace}
\newcommand{\obda}{o.\,B.\,d.\,A.\xspace}

\lstset{% general command to set parameter(s)
  language=c++,
   basicstyle=\footnotesize\ttfamily,          % print whole listing small
   keywordstyle=\color{DodgerBlue4}\bfseries\underbar,
                               % underlined bold black keywords
   identifierstyle=\color{black},           % nothing happens
   commentstyle=\bfseries\color{Azure4}, % white comments
   stringstyle=\underbar\itshape\color{IndianRed2},      % typewriter type for strings
   showstringspaces=false}     % no special string spaces


% f"ur einheitlichen code-includings: z.B. codefile{hello_world.cpp,Legend"ares \dq Hello World \dq example}
\newcommand{\codefile}[2]{\lstinputlisting[numbers=left,stringstyle=\texttt,label=#1,caption=#2]{examples/#1}}
%\newcommand{\coderow}[1]{
%  \marginpar{
%    \begin{lstlisting}
%      {\small #1} 
%    \end{lstlisting}
%  }
%}


%noch nicht feddig
%\newenvironment{code}[2]{\begin{lstlisting}[label=#1,caption=#2]}{\end{lstlisting}}


